\documentclass[a4paper,14pt]{extarticle}

\usepackage[utf8]{inputenc}
\usepackage[russian]{babel}
\usepackage{graphicx}
\usepackage[top=0.8in, bottom=0.8in, left=0.8in, right=0.8in]{geometry}
\usepackage{pgfplots}
\usepackage{amsmath}
\usepackage{setspace}
\usepackage{titlesec}
\usepackage{float}
\usepackage{chngcntr}
\usepackage{pgfplots}
\usepackage{amsfonts}
\usepackage{pgfplotstable}
\usepackage{multirow}
\usepackage{karnaugh-map}
\usepackage{tikz,xcolor}
\usepackage{listings}
\usepackage{indentfirst}
\usepackage{hyperref}

\hypersetup{
    colorlinks=true,
    linkcolor=blue,
    filecolor=magenta,      
    urlcolor=cyan,
}

\lstset{ %
extendedchars=\true,
keepspaces=true,
%language=no, % choose the language of the code
basicstyle=\footnotesize, % the size of the fonts that are used for the code
numbers=left, % where to put the line-numbers
numberstyle=\footnotesize, % the size of the fonts that are used for the line-numbers
stepnumber=1, % the step between two line-numbers. If it is 1 each line will be numbered
numbersep=10pt, % how far the line-numbers are from the code
backgroundcolor=\color{white}, % choose the background color. You must add \usepackage{color}
showspaces=false, % show spaces adding particular underscores
showstringspaces=false, % underline spaces within strings
showtabs=false, % show tabs within strings adding particular underscores
frame=single, % adds a frame around the code
tabsize=4, % sets default tabsize to 2 spaces
captionpos=b, % sets the caption-position to bottom
breaklines=true, % sets automatic line breaking
breakatwhitespace=false, % sets if automatic breaks should only happen at whitespace
escapeinside={\%*}{*)}, % if you want to add a comment within your code
%postbreak=\raisebox{0ex}[0ex][0ex]{\ensuremath{\color{red}\hookrightarrow\space}}
}

\titleformat{\section}[hang]
  {\bfseries}
  {}
  {0em}
  {\hspace{-0.4pt}\large \thesection\hspace{0.6em}}
  
  
\titleformat{\subsection}[hang]
  {\bfseries}
  {}
  {0em}
  {\hspace{-0.4pt}\large \thesubsection\hspace{0.6em}}

%\linespread{1.3} % полуторный интервал
%\renewcommand{\rmdefault}{ftm} % Times New Roman

\newcommand{\nx}{\overline{x}}
\newcommand{\p}{0.31}
\newcommand{\scale}{1.4}

\counterwithin{figure}{section}
\counterwithin{equation}{section}
\counterwithin{table}{section}



\begin{document}

%
\begin{titlepage}
\centering
Санкт-Петербургский политехнический университет Петра Великого \\
\vspace{0.15cm}
Кафедра компьютерных систем и программных технологий \\
\vspace{6.5cm}

{\centering \textbf{Отчёт по лабораторной работе} \\ 
\vspace{0.15cm}
\textbf{Дисциплина}: Транслирующие системы \\
\vspace{0.15cm}
\textbf{Тема}: Транслятор операторов $while$ языка $C$ } \\

\vspace{6.5cm}

\begin{table}[H]
\begin{tabular}{p{\textwidth}@{}r}
{Выполнил студент гр. 43501/3} \hfill {Мальцев  М.С.} \\
{Преподаватель} \hfill {Цыган В.Н.} \\
\end{tabular}
\end{table}
\vfill

{\centering Санкт-Петербург \\ 
\vspace{0.15cm}
\today}
\end{titlepage}

\section{Введение}
	Корабли бороздят просторы океана...
	
	Но по сути ничего не меняется, реальность циклична.

	Технологии развиваются и если не выйдет еще один js-фреймворк, то
	точно произойдет конец света. Каждый новый приносит какую-то
	новую фишку.

	Куча мобильных устройств и нужно чтобы пользователи получали удовольствие
	от использования сайтиков с них.
	Соответственно мир меняется. Меняются требования и запросы.

	Существует шаблоны формирования графического интерфейса, например, материал дизайн.
	И лучше им следовать.

	Очень часто у людей становится вопрос "Как изучить английский язык?"{} и если
	с изучением правил грамматики и лексики обычно проблем не возникает, то с произношением
	всё обстоит гораздо сложнее.
	На практике лучший вариант изучить язык - это оказаться в среде, в которой общество вокруг человека
	использует этот язык.
	Но не у всех есть такая возможность. Существует множество всевозможных приложений, которые
	нацелены на увеличение словарного запаса и на понимание грамматики языка.
	Часто становится проблема, заключающаяся в том, что человек произносит слова не правильно или
	у него очень сильный акцент. 

	Разрабатываемое приложение позволяет создавать и редактировать курсы для изучения языков с
	акцентом на интонации.
	Считается, что интонации являются очень важной частью языка.





	%%%%%%%%%%%%%%%%%%%%%%%%

	Во введении обосновывается актуальность темы магистерской диссертации,
	выявляются противоречия, определяется ее цель, формулируются задачи, которые
	необходимо решить для достижения поставленной цели, дается характеристика
	разработанности проблемы, указывается методология и методика исследования, дается
	характеристика эмпирической и источниковой базы исследования, презентуется его
	новизна

\section{Анализ}
	Основная часть работы включает два- три раздела или главы, которые
	разбивают на подразделы. Каждый раздел (подраздел) посвящен решению задач,
	сформулированных во введении, и заканчивается выводами, к которым пришел
	магистрант в результате проведенных исследований. Названия глав должны быть
	предельно краткими, четкими, точно отражать их основное содержание и не могут
	повторять название диссертации.

\section{Проектирование}

	Было решено использовать классный стек.
	Стек называется MERN - MongoDB, Express, React, NodeJs.
	
	Стек был выбран, потому что он элегантный и релевантный, а также потому что
	хотелось попробовать что-то новое в место стандартных и приевшихся.
	
	Помимо NodeJs - Java, PHP, C\#, C++.

	Вместо MongoDB - NoSQL подходы  такие как CouchDB, а в случае SQL подхода
	MySQl, PostgresSQL, MariaDB, Oracl.

	Express - в случае, если мы уже выбрали NodeJs, можно заменить Koa.

	React - заменяется либо нативным решением (т.е. решением на основе HTML, CSS, JS,
	без использования фреймворков), либо другим фреймворком, таким как View, Angular.

	MongoDB позволяет эффективно строить распределенные системы, нам это не надо, но
	звучит довольно классно.

	NodeJs хорош тем, что используя этот язык мы получаем один и тот же язык во фронтенде
	и в бэкенде, что упрощает разработку.
	
	Разарботка серверной части на Express одно удовольствие (ну норм же тейк).

	React был выбран потому что есть такая шикарная технология, как ReactNative.
	Также потому что я знаю только его. Точнее и его не знаю, но лейбл красивый.


 	
\section{Разработка}
	Для разработки использовались многие интересные штуки,
	например, ES Lint, nodeamon.

	React делался отдельно потом прикручивался.

	Разработка приложения получилось довольно долгой, потому что была куча вещей, на которую
	можно было успешно отвлечься.


 	
\section{Тестирование}
	Хочется зачётик поэтому пришлось тестировать.
		 
\section{Заключение}
	Заключение должно быть прямо связано с теми целями и задачами, которые
	сформулированы во введении. Здесь даются выводы и обобщения, вытекающие из всей 
	работы, даются рекомендации, указываются пути дальнейших исследований в рамках
	данной проблемы.

\section{Список используемых источников}
	Половина из головы, половина из других странных источников.

\end{document} 
